\section{旅行指南} % (fold)
\label{sec:旅行指南}

Linux 系统中,文件系统布局与类似 Unix 系统的文件布局很相似。实际上,一个已经发布的标准, 叫做 Linux 文件系统层次标准,详细说明了这种设计模式。不是所有Linux发行版都根据这个标准,但 大多数都是。

\par 下一步,我们将在文件系统中游玩,来了解 Linux 系统的工作原理。这会给你一个温习跳转命令的机会。 我们会发现很多有趣的文件都是普通的可读文本。将开始旅行,做做以下练习:
\begin{enumerate}
	\item cd 到给定目录
	\item 列出目录内容 ls -l
	\item 如果看到一个有趣的文件,用 file 命令确定文件内容
	\item 如果文件看起来像文本,试着用 less 命令浏览它
\end{enumerate}
\fboxrule=3pt \fboxsep=2pt
\begin{colorboxed}[boxcolor=lightgray,bgcolor=white]
\textbf{记得复制和粘贴技巧!}如果你正在使用鼠标,双击文件名,来复制它,然后按下鼠标中键,粘贴文件名到命令行中。
\end{colorboxed}

\par 在系统中游玩时,不要害怕粘花惹草。普通用户是很难把东西弄乱的。那是系统管理员的工作! 如果一个命令抱怨一些事情,不要管它,尽管去玩别的东西。花一些时间四处走走。 系统是我们自己的,尽情地探究吧。记住在 Linux 中,没有秘密存在! 表3-4仅仅列出了一些我们可以浏览的目录。闲暇时试试看!


% section 旅行指南 (end)

