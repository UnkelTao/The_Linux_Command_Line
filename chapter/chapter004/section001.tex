\section{通配符} % (fold)
\label{sec:通配符}
在开始使用命令之前,我们需要介绍一个使命令行如此强大的 shell 特性。因为 shell 频繁地使用 文件名,shell 提供了特殊字符来帮助你快速指定一组文件名。这些特殊字符叫做通配符。 使用通配符(也以文件名代换著称)允许你依据字符类型来选择文件名。下表列出这些通配符 以及它们所选择的对象:

\begin{center} \tablecaption{通配符 \label{tab:super1}}
\tablefirsthead{%
\rowcolor[gray]{0.8}
\multicolumn{1}{l}{\textbf{通配符}} & 
\multicolumn{1}{c}{\textbf{意义}} \\ }
\tablehead{\multicolumn{2}{c}{%
\small 表 \ref{tab:super1} (续) } \\
\rowcolor[gray]{0.8}
\multicolumn{1}{l}{\textbf{通配符}} & 
\multicolumn{1}{c}{\textbf{意义}} \\}
\tabletail{\bottomrule
\multicolumn{2}{c}{\small 接下页} \\
}
\tablelasttail{\bottomrule}

\begin{supertabular}{p{4.cm}p{10cm}}
\verb"*"	& 匹配任意多个字符(包括零个或一个) \\
\verb"?"	& 匹配任意一个字符(不包括零个)\\
\verb"["characters\verb"]"	& 匹配任意一个属于字符集中的字符 \\
\verb"["\verb"!"characters\verb"]"	& 匹配任意一个不是字符集中的字符 \\
\verb"["\verb"[":class:\verb"]"\verb"]"	& 匹配任意一个属于指定字符类中的字符 \\

\end{supertabular}
\end{center}

\par 表\ref{tab:super2}列出了最常使用的字符类:

\begin{center} \tablecaption{通配符 \label{tab:super2}}
\tablefirsthead{%
\rowcolor[gray]{0.8}
\multicolumn{1}{l}{\textbf{字符类}} & 
\multicolumn{1}{c}{\textbf{意义}} \\ }
\tablehead{\multicolumn{2}{c}{%
\small 表 \ref{tab:super2} (续) } \\
\rowcolor[gray]{0.8}
\multicolumn{1}{l}{\textbf{字符类}} & 
\multicolumn{1}{c}{\textbf{意义}} \\}
\tabletail{\bottomrule
\multicolumn{2}{c}{\small 接下页} \\
}
\tablelasttail{\bottomrule}

\begin{supertabular}{p{4.cm}p{10cm}}
\verb"["\verb":"alnum\verb":"\verb"]"	& 匹配任意一个字母或数字 \\
\verb"["\verb":"alpha\verb":"\verb"]" & 匹配任意一个字母 \\
\verb"["\verb":"digit\verb":"\verb"]"	& 匹配任意一个数字 \\
\verb"["\verb":"lower\verb":"\verb"]"	& 匹配任意一个小写字母 \\
\verb"["\verb":"upper\verb":"\verb"]"	& 匹配任意一个大写字母 \\

\end{supertabular}
\end{center}

\par 借助通配符,为文件名构建非常复杂的选择标准成为可能。下面是一些类型匹配的范例:
\begin{center} \tablecaption{通配符范例 \label{tab:super3}}
\tablefirsthead{%
\rowcolor[gray]{0.8}
\multicolumn{1}{l}{\textbf{模式}} & 
\multicolumn{1}{c}{\textbf{匹配对象}} \\ }
\tablehead{\multicolumn{2}{c}{%
\small 表 \ref{tab:super3} (续) } \\
\rowcolor[gray]{0.8}
\multicolumn{1}{l}{\textbf{模式}} & 
\multicolumn{1}{c}{\textbf{匹配对象}} \\}
\tabletail{\bottomrule
\multicolumn{2}{c}{\small 接下页} \\
}
\tablelasttail{\bottomrule}

\begin{supertabular}{p{5.cm}p{11cm}}
\verb"*"	& 所有文件 \\
g*	& 文件名以“g”开头的文件 \\
b*.txt &	以"b"开头,中间有零个或任意多个字符,并以".txt"结尾的文件 \\
Data???	& 以``Data''开头,其后紧接着3个字符的文件 \\
\verb"["abc\verb"]"*	& 文件名以"a","b",或"c"开头的文件 \\
BACKUP.\verb"["0-9][0-9][0-9]	& 以"BACKUP."开头,并紧接着3个数字的文件 \\
\verb"["\verb"["\verb":"upper\verb":"\verb"]"\verb"]"* & 以大写字母开头的文件 \\
\verb"["\verb"!"\verb"["\verb":"digit\verb":"\verb"]"\verb"]"* & 不以数字开头的文件 \\
\verb"*"\verb"["\verb"["\verb":"lower\verb":"\verb"]"123\verb"]"	& 文件名以小写字母结尾,或以 “1”,“2”,或 “3” 结尾的文件 \\

\end{supertabular}
\end{center}

接受文件名作为参数的任何命令,都可以使用通配符,我们会在第八章更深入的谈到这个知识点。

\fboxrule=6pt \fboxsep=4pt
\begin{colorboxed}[boxcolor=lightgray,bgcolor=white]
\subsection{字符范围}
如果你用过别的类似 Unix 系统的操作环境,或者是读过这方面的书籍,你可能遇到过[A-Z]或 [a-z]形式的字符范围表示法。这些都是传统的 Unix 表示法,并且在早期的 Linux 版本中仍有效。 虽然它们仍然起作用,但是你必须小心地使用它们,因为它们不会产生你期望的输出结果,除非 你合理地配置它们。从现在开始,你应该避免使用它们,并且用字符类来代替它们。

\subsection{通配符在 GUI 中也有效}
通配符非常重要,不仅因为它们经常用在命令行中,而且一些图形文件管理器也支持它们。
\begin{itemize}
	\item 在 Nautilus (GNOME 文件管理器)中,可以通过 Edit/Select 模式菜单项来选择文件。 输入一个用通配符表示的文件选择模式后,那么当前所浏览的目录中,所匹配的文件名 就会高亮显示。
	\item 在 Dolphin 和 Konqueror(KDE 文件管理器)中,可以在地址栏中直接输入通配符。例如,如果你 想查看目录 /usr/bin 中,所有以小写字母 ``u'' 开头的文件,在地址栏中敲入 ``/usr/bin/u*'',则 文件管理器会显示匹配的结果。
\end{itemize}


\par 最初源于命令行界面中的想法,在图形界面中也适用。这就是使 Linux 桌面系统 如此强大的众多原因中的一个。
\end{colorboxed}

% section 通配符 (end)