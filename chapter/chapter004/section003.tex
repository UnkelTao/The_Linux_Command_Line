\section{cp — 复制文件和目录} % (fold)
\label{sec:cp_复制文件和目录}

cp 命令,复制文件或者目录。它有两种使用方法:
\begin{lstlisting}
cp item1 item2
\end{lstlisting}
\par 复制单个文件或目录”item1”到文件或目录”item2”,和:
\begin{lstlisting}
cp item... directory
\end{lstlisting}
\par 复制多个项目(文件或目录)到一个目录下。

\subsection{有用的选项和实例}
这里列举了 cp 命令一些有用的选项(短选项和等效的长选项):

\begin{center} \tablecaption{CP选项 \label{tab:super4}}
\tablefirsthead{%
\rowcolor[gray]{0.8}
\multicolumn{1}{l}{\textbf{选项}} & 
\multicolumn{1}{c}{\textbf{意义}} \\ }
\tablehead{\multicolumn{2}{c}{%
\small 表 \ref{tab:super4} (续) } \\
\rowcolor[gray]{0.8}
\multicolumn{1}{l}{\textbf{选项}} & 
\multicolumn{1}{c}{\textbf{意义}} \\}
\tabletail{\bottomrule
\multicolumn{2}{c}{\small 接下页} \\
}
\tablelasttail{\bottomrule}

\begin{supertabular}{p{3.5cm}p{10cm}}
-a, -\-archive	& 复制文件和目录,以及它们的属性,包括所有权和权限。 通常,复本具有用户所操作文件的默认属性。\\
-i, -\-interactive & 	在重写已存在文件之前,提示用户确认。如果这个选项不指定, cp 命令会默认重写文件。\\
-r, -\-recursive	& 递归地复制目录及目录中的内容。当复制目录时, 需要这个选项(或者-a 选项)。\\
-u, -\-update	& 当把文件从一个目录复制到另一个目录时,仅复制 目标目录中不存在的文件,或者是文件内容新于目标目录中已经存在的文件。\\
-v, -\-verbose	& 显示翔实的命令操作信息 \\

\end{supertabular}
\end{center}


\begin{center} \tablecaption{CP实例 \label{tab:super5}}
\tablefirsthead{%
\rowcolor[gray]{0.8}
\multicolumn{1}{l}{\textbf{命令}} & 
\multicolumn{1}{c}{\textbf{运行结果}} \\ }
\tablehead{\multicolumn{2}{c}{%
\small 表 \ref{tab:super5} (续) } \\
\rowcolor[gray]{0.8}
\multicolumn{1}{l}{\textbf{命令}} & 
\multicolumn{1}{c}{\textbf{运行结果}} \\}
\tabletail{\bottomrule
\multicolumn{2}{c}{\small 接下页} \\
}
\tablelasttail{\bottomrule}

\begin{supertabular}{p{3.5cm}p{10cm}}
cp file1 file2 & 复制文件 file1内容到文件file2。如果 file2已经存在,file2的内容会被 file1的 内容重写。如果 file2不存在,则会创建 file2。\\
cp -i file1 file2	& 这条命令和上面的命令一样,除了如果文件 file2存在的话,在文件 file2被重写之前, 会提示用户确认信息。\\
cp file1 file2 dir1	& 复制文件 file1和文件 file2到目录 dir1。目录 dir1必须存在。\\
cp dir1/* dir2	& 使用一个通配符,在目录 dir1中的所有文件都被复制到目录 dir2中。 dir2必须已经存在。\\
cp -r dir1 dir2 & 复制目录 dir1中的内容到目录 dir2。如果目录 dir2不存在, 创建目录 dir2,操作完成后,目录 dir2中的内容和 dir1中的一样。 如果目录 dir2存在,则目录 dir1(和目录中的内容)将会被复制到 dir2中。\\

\end{supertabular}
\end{center}
% section cp_复制文件和目录 (end)