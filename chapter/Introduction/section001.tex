\section{为什么使用命令行}
\label{为什么使用命令行}

你是否注意到,在电影中一个“超级黑客”坐在电脑前,从不摸一下鼠标, 就能够在30秒内侵入到超安全的军事计算机中。这是因为电影制片人意识到, 作为人类,我们应该本能地知道让计算机圆满完成工作的唯一途径,是用键盘来操纵计算机。

\par 现在,大多数的计算机用户只是熟悉图形用户界面(GUI),并且产品供应商和此领域的学者会灌输给用户这样的思想, 命令行界面(CLI)是过去使用的一种很恐怖的东西。这就很不幸,因为一个友好的命令行界面, 是用来和计算机进行交流沟通的,正像人类社会使用文字互通信息一样。据说,“图形用户界面让简单的任务更容易完成, 而命令行界面使完成复杂的任务成为可能”,到现在这句话仍然很正确。

\par 因为 Linux 是以 Unix 家族的操作系统为模型写成的,所以它分享了 Unix 丰富的命令行工具。 Unix 在20世纪80年代初显赫一时(虽然,开发它在更早之前),结果,在普遍地使用图形界面之前, 开发了一种广泛的命令行界面。事实上,一个主要的原因,Linux 开发者优先采用命令行界面 而不是其他的系统,比如说 Windows NT,是因为其强大的命令行界面,可以使“完成复杂的任务成为可能”。

% section 为什么使用命令行 (end)