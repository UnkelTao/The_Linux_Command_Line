\section{怎样阅读这本书} % (fold)
\label{sec:怎样阅读这本书}

从头到尾的阅读。它并不是一本技术参考手册,实际上它更像一本故事书,有开头,过程,结尾。

\subsection{前提条件} % (fold)
\label{ssub:_前提条件}
为了使用这本书,你需要安装 Linux 操作系统。你可以通过两种方式,来完成安装。
\begin{enumerate}
\item  在一台(不是很新)的电脑上安装 Linux。你选择哪个 Linux 发行版安装,是无关紧要的事。 虽然大多数人一开始选择安装 Ubuntu, Fedora, 或者 OpenSUSE。如果你拿不定主意,那就先试试 Ubuntu。 由于主机硬件配置不同,安装 Linux 时,你可能不费吹灰之力就装上了,也可能费了九牛二虎之力还装不上。 所以我建议,一台使用了几年的台式机,至少要有256M 的内存,6G 的硬盘可用空间。尽可能避免使用 笔记本电脑和无线网络,在 Linux 环境下,它们经常不能工作。

\item  使用“Live CD.” 许多 Linux 发行版都自带一个比较酷的功能,你可以直接从系统安装盘 CDROM 中运行 Linux, 而不必安装 Linux。开机进入 BIOS 设置界面,更改引导项,设置为“从 CDROM 启动”。
\end{enumerate}

\par 不管你怎样安装 Linux,为了练习书中介绍的知识,你需要有超级用户(管理员)权限。

\par 当你在自己的电脑上安装了 Linux 系统之后,就开始一边阅读本书,一边练习吧。本书大部分内容 都可以自己动手练习,坐下来,敲入命令,体验一下吧。



%\lemmabox{

\fboxrule=6pt \fboxsep=4pt
\begin{colorboxed}[boxcolor=lightgray,bgcolor=white]
 \subsubsection{为什么我不叫它“GNU/Linux”}

\par 在某些领域,把 Linux 操作系统称为“GNU/Linux 操作系统.”是比较明智的做法。但“Linux”的问题在于, 没有一个完全正确的方式能为它命名,因为它是由许许多多,分布在世界各地的贡献者们,合作开发而成的。 从技术层面讲,Linux 只是操作系统的内核名字,没别的含义。当然内核非常重要,有了内核, 操作系统才能运行起来,但它并不能构成一个完整的操作系统。

\par Richard Stallman 是一个天才的哲学家,自由软件运动创始人,自由软件基金会创办者,他创建了 GNU 工程, 编写了第一版 GNU C 编译器(gcc),创立了 GNU 通用公共协议(the GPL)等等。 他坚持把 Linux 称为“GNU/Linux”,为的是准确地反映 GNU 工程对 Linux 操作系统的贡献。 然而,GNU 项目早于 Linux 内核,而 GNU 项目的贡献得到了极高的赞誉,再把 GNU 用在 Linux 名字里, 这对其他每个为 Linux 的发展做出重大贡献的程序员来说,就不公平了。

\par 在目前流行的用法中,“Linux”指的是内核以及在一个典型的 Linux 发行版中所包含的所有免费及开源软件; 也就是说,整个 Linux 生态系统,不只有 GNU 项目软件。在操作系统商界,好像喜欢使用单个词的名字, 比如说 DOS, Windows, MacOS, Solaris, Irix, AIX. 所以我选择用流行的命名规则。然而, 如果你喜欢用“GNU/Linux”,当你读这本书时,可以搜索并代替“Linux”。我不介意。
\end{colorboxed}


% subsection _前提条件 (end)
% section 怎样阅读这本书 (end)