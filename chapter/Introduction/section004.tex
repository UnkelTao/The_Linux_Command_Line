\section{这本书的内容} % (fold)
\label{sec:这本书的内容}

这些材料是经过精心安排的,很像一位老师坐在你身旁,耐心地指导你。 许多作者用系统化的方式讲解这些材料,虽然从一个作者的角度考虑很有道理,但对于 Linux 新手来说, 他们可能会感到非常困惑。

\par 另一个目的,是想让读者熟悉 Unix 的思维方式,这种思维方式不同于 Windows 的。在学习过程中, 我们会帮助你理解为什么某些命令会按照它们的方式工作,以及它们是怎样实现那样的工作方式的。 Linux 不仅是一款软件,也是 Unix 文化的一小部分,它有自己的语言和历史渊源。 同时,我也许会说些过激的话。

\par 这本书共分为五部分,每一部分讲述了不同方面的命令行知识。除了第一部分, 也就是你正在阅读的这一部分,这本书还包括:
\begin{itemize}
	\item 第二部分 — 学习 shell 开始探究命令行基本语言,包括命令组成结构, 文件系统浏览,编写命令行,查找命令帮助文档。
	\item 第三部分 — 配置文件及环境 讲述了如何编写配置文件,通过配置文件,用命令行来 操控计算机。
	\item 第四部分 — 常见任务及主要工具 探究了许多命令行经常执行的普通任务。类似于 Unix 的操作系统,例如 Linux, 包括许多经典的命令行程序,这些程序可以用来对数据进行 强大的操作。
	\item 第五部分 — 编写 Shell 脚本 介绍了 shell 编程,一个无可否认的基本技能,能够自动化许多 常见的计算任务,很容易学。通过学习 shell 编程,你会逐渐熟悉一些关于编程语言方面的概念, 这些概念也适用于其他的编程语言。
\end{itemize}





% section 这本书的内容 (end)