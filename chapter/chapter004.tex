\renewcommand\chapterillustration{cherry-tomatos}
\chapter{操作文件和目录}

此时此刻,我们已经准备好了做些真正的工作!这一章节将会介绍以下命令:

\begin{itemize}
	\item cp — 复制文件和目录
	\item mv — 移动/重命名文件和目录
	\item mkdir — 创建目录
	\item rm — 删除文件和目录
	\item ln — 创建硬链接和符号链接
\end{itemize}

\par 这五个命令属于最常使用的 Linux 命令之列。它们用来操作文件和目录。

\par 现在,坦诚地说,用图形文件管理器来完成一些由这些命令执行的任务会更容易些。使用文件管理器, 我们可以把文件从一个目录拖放到另一个目录,剪贴和粘贴文件,删除文件等等。那么, 为什么还使用早期的命令行程序呢?

\par 答案是命令行程序,功能强大灵活。虽然图形文件管理器能轻松地实现简单的文件操作,但是对于 复杂的文件操作任务,则使用命令行程序比较容易完成。例如,怎样复制一个目录下的 HTML 文件 到另一个目录,但这些 HTML 文件不存在于目标目录,或者是文件版本新于目标目录里的文件? 要完成这个任务,使用文件管理器相当难,使用命令行相当容易:

\begin{lstlisting}
cp -u *.html destination 
\end{lstlisting}

\section{通配符} % (fold)
\label{sec:通配符}
在开始使用命令之前,我们需要介绍一个使命令行如此强大的 shell 特性。因为 shell 频繁地使用 文件名,shell 提供了特殊字符来帮助你快速指定一组文件名。这些特殊字符叫做通配符。 使用通配符(也以文件名代换著称)允许你依据字符类型来选择文件名。下表列出这些通配符 以及它们所选择的对象:

\begin{center} \tablecaption{通配符 \label{tab:super1}}
\tablefirsthead{%
\rowcolor[gray]{0.8}
\multicolumn{1}{l}{\textbf{通配符}} & 
\multicolumn{1}{c}{\textbf{意义}} \\ }
\tablehead{\multicolumn{2}{c}{%
\small 表 \ref{tab:super1} (续) } \\
\rowcolor[gray]{0.8}
\multicolumn{1}{l}{\textbf{通配符}} & 
\multicolumn{1}{c}{\textbf{意义}} \\}
\tabletail{\bottomrule
\multicolumn{2}{c}{\small 接下页} \\
}
\tablelasttail{\bottomrule}

\begin{supertabular}{p{4.cm}p{10cm}}
\verb"*"	& 匹配任意多个字符(包括零个或一个) \\
\verb"?"	& 匹配任意一个字符(不包括零个)\\
\verb"["characters\verb"]"	& 匹配任意一个属于字符集中的字符 \\
\verb"["\verb"!"characters\verb"]"	& 匹配任意一个不是字符集中的字符 \\
\verb"["\verb"[":class:\verb"]"\verb"]"	& 匹配任意一个属于指定字符类中的字符 \\

\end{supertabular}
\end{center}

\par 表\ref{tab:super2}列出了最常使用的字符类:

\begin{center} \tablecaption{通配符 \label{tab:super2}}
\tablefirsthead{%
\rowcolor[gray]{0.8}
\multicolumn{1}{l}{\textbf{字符类}} & 
\multicolumn{1}{c}{\textbf{意义}} \\ }
\tablehead{\multicolumn{2}{c}{%
\small 表 \ref{tab:super2} (续) } \\
\rowcolor[gray]{0.8}
\multicolumn{1}{l}{\textbf{字符类}} & 
\multicolumn{1}{c}{\textbf{意义}} \\}
\tabletail{\bottomrule
\multicolumn{2}{c}{\small 接下页} \\
}
\tablelasttail{\bottomrule}

\begin{supertabular}{p{4.cm}p{10cm}}
\verb"["\verb":"alnum\verb":"\verb"]"	& 匹配任意一个字母或数字 \\
\verb"["\verb":"alpha\verb":"\verb"]" & 匹配任意一个字母 \\
\verb"["\verb":"digit\verb":"\verb"]"	& 匹配任意一个数字 \\
\verb"["\verb":"lower\verb":"\verb"]"	& 匹配任意一个小写字母 \\
\verb"["\verb":"upper\verb":"\verb"]"	& 匹配任意一个大写字母 \\

\end{supertabular}
\end{center}

\par 借助通配符,为文件名构建非常复杂的选择标准成为可能。下面是一些类型匹配的范例:
\begin{center} \tablecaption{通配符范例 \label{tab:super3}}
\tablefirsthead{%
\rowcolor[gray]{0.8}
\multicolumn{1}{l}{\textbf{模式}} & 
\multicolumn{1}{c}{\textbf{匹配对象}} \\ }
\tablehead{\multicolumn{2}{c}{%
\small 表 \ref{tab:super3} (续) } \\
\rowcolor[gray]{0.8}
\multicolumn{1}{l}{\textbf{模式}} & 
\multicolumn{1}{c}{\textbf{匹配对象}} \\}
\tabletail{\bottomrule
\multicolumn{2}{c}{\small 接下页} \\
}
\tablelasttail{\bottomrule}

\begin{supertabular}{p{5.cm}p{11cm}}
\verb"*"	& 所有文件 \\
g*	& 文件名以“g”开头的文件 \\
b*.txt &	以"b"开头,中间有零个或任意多个字符,并以".txt"结尾的文件 \\
Data???	& 以``Data''开头,其后紧接着3个字符的文件 \\
\verb"["abc\verb"]"*	& 文件名以"a","b",或"c"开头的文件 \\
BACKUP.\verb"["0-9][0-9][0-9]	& 以"BACKUP."开头,并紧接着3个数字的文件 \\
\verb"["\verb"["\verb":"upper\verb":"\verb"]"\verb"]"* & 以大写字母开头的文件 \\
\verb"["\verb"!"\verb"["\verb":"digit\verb":"\verb"]"\verb"]"* & 不以数字开头的文件 \\
\verb"*"\verb"["\verb"["\verb":"lower\verb":"\verb"]"123\verb"]"	& 文件名以小写字母结尾,或以 “1”,“2”,或 “3” 结尾的文件 \\

\end{supertabular}
\end{center}

接受文件名作为参数的任何命令,都可以使用通配符,我们会在第八章更深入的谈到这个知识点。

\fboxrule=6pt \fboxsep=4pt
\begin{colorboxed}[boxcolor=lightgray,bgcolor=white]
\subsection{字符范围}
如果你用过别的类似 Unix 系统的操作环境,或者是读过这方面的书籍,你可能遇到过[A-Z]或 [a-z]形式的字符范围表示法。这些都是传统的 Unix 表示法,并且在早期的 Linux 版本中仍有效。 虽然它们仍然起作用,但是你必须小心地使用它们,因为它们不会产生你期望的输出结果,除非 你合理地配置它们。从现在开始,你应该避免使用它们,并且用字符类来代替它们。

\subsection{通配符在 GUI 中也有效}
通配符非常重要,不仅因为它们经常用在命令行中,而且一些图形文件管理器也支持它们。
\begin{itemize}
	\item 在 Nautilus (GNOME 文件管理器)中,可以通过 Edit/Select 模式菜单项来选择文件。 输入一个用通配符表示的文件选择模式后,那么当前所浏览的目录中,所匹配的文件名 就会高亮显示。
	\item 在 Dolphin 和 Konqueror(KDE 文件管理器)中,可以在地址栏中直接输入通配符。例如,如果你 想查看目录 /usr/bin 中,所有以小写字母 ``u'' 开头的文件,在地址栏中敲入 ``/usr/bin/u*'',则 文件管理器会显示匹配的结果。
\end{itemize}


\par 最初源于命令行界面中的想法,在图形界面中也适用。这就是使 Linux 桌面系统 如此强大的众多原因中的一个。
\end{colorboxed}

% section 通配符 (end)

\section{mkdir - 创建文件夹} % (fold)
\label{sec:mkdir_创建文件夹}
mkdir 命令是用来创建目录的。它这样工作:
\begin{lstlisting}
mkdir directory...
\end{lstlisting}
\par \textbf{注意表示法}: 在描述一个命令时(如上所示),当有三个圆点跟在一个命令的参数后面, 这意味着那个参数可以重复,就像这样:
\begin{lstlisting}
mkdir dir1
\end{lstlisting}
\par 会创建一个名为”dir1”的目录,而
\begin{lstlisting}
mkdir dir1 dir2 dir3
\end{lstlisting}
\par 会创建三个目录,名为``dir1'', ``dir2'', ``dir3''。


% section mkdir_创建文件夹 (end)

\section{cp — 复制文件和目录} % (fold)
\label{sec:cp_复制文件和目录}

cp 命令,复制文件或者目录。它有两种使用方法:
\begin{lstlisting}
cp item1 item2
\end{lstlisting}
\par 复制单个文件或目录”item1”到文件或目录”item2”,和:
\begin{lstlisting}
cp item... directory
\end{lstlisting}
\par 复制多个项目(文件或目录)到一个目录下。

\subsection{有用的选项和实例}
这里列举了 cp 命令一些有用的选项(短选项和等效的长选项):

\begin{center} \tablecaption{CP选项 \label{tab:super4}}
\tablefirsthead{%
\rowcolor[gray]{0.8}
\multicolumn{1}{l}{\textbf{选项}} & 
\multicolumn{1}{c}{\textbf{意义}} \\ }
\tablehead{\multicolumn{2}{c}{%
\small 表 \ref{tab:super4} (续) } \\
\rowcolor[gray]{0.8}
\multicolumn{1}{l}{\textbf{选项}} & 
\multicolumn{1}{c}{\textbf{意义}} \\}
\tabletail{\bottomrule
\multicolumn{2}{c}{\small 接下页} \\
}
\tablelasttail{\bottomrule}

\begin{supertabular}{p{3.5cm}p{10cm}}
-a, -\-archive	& 复制文件和目录,以及它们的属性,包括所有权和权限。 通常,复本具有用户所操作文件的默认属性。\\
-i, -\-interactive & 	在重写已存在文件之前,提示用户确认。如果这个选项不指定, cp 命令会默认重写文件。\\
-r, -\-recursive	& 递归地复制目录及目录中的内容。当复制目录时, 需要这个选项(或者-a 选项)。\\
-u, -\-update	& 当把文件从一个目录复制到另一个目录时,仅复制 目标目录中不存在的文件,或者是文件内容新于目标目录中已经存在的文件。\\
-v, -\-verbose	& 显示翔实的命令操作信息 \\

\end{supertabular}
\end{center}


\begin{center} \tablecaption{CP实例 \label{tab:super5}}
\tablefirsthead{%
\rowcolor[gray]{0.8}
\multicolumn{1}{l}{\textbf{命令}} & 
\multicolumn{1}{c}{\textbf{运行结果}} \\ }
\tablehead{\multicolumn{2}{c}{%
\small 表 \ref{tab:super5} (续) } \\
\rowcolor[gray]{0.8}
\multicolumn{1}{l}{\textbf{命令}} & 
\multicolumn{1}{c}{\textbf{运行结果}} \\}
\tabletail{\bottomrule
\multicolumn{2}{c}{\small 接下页} \\
}
\tablelasttail{\bottomrule}

\begin{supertabular}{p{3.5cm}p{10cm}}
cp file1 file2 & 复制文件 file1内容到文件file2。如果 file2已经存在,file2的内容会被 file1的 内容重写。如果 file2不存在,则会创建 file2。\\
cp -i file1 file2	& 这条命令和上面的命令一样,除了如果文件 file2存在的话,在文件 file2被重写之前, 会提示用户确认信息。\\
cp file1 file2 dir1	& 复制文件 file1和文件 file2到目录 dir1。目录 dir1必须存在。\\
cp dir1/* dir2	& 使用一个通配符,在目录 dir1中的所有文件都被复制到目录 dir2中。 dir2必须已经存在。\\
cp -r dir1 dir2 & 复制目录 dir1中的内容到目录 dir2。如果目录 dir2不存在, 创建目录 dir2,操作完成后,目录 dir2中的内容和 dir1中的一样。 如果目录 dir2存在,则目录 dir1(和目录中的内容)将会被复制到 dir2中。\\

\end{supertabular}
\end{center}
% section cp_复制文件和目录 (end)


% \section{旅行指南} % (fold)
\label{sec:旅行指南}

Linux 系统中,文件系统布局与类似 Unix 系统的文件布局很相似。实际上,一个已经发布的标准, 叫做 Linux 文件系统层次标准,详细说明了这种设计模式。不是所有Linux发行版都根据这个标准,但 大多数都是。

\par 下一步,我们将在文件系统中游玩,来了解 Linux 系统的工作原理。这会给你一个温习跳转命令的机会。 我们会发现很多有趣的文件都是普通的可读文本。将开始旅行,做做以下练习:
\begin{enumerate} 
	\item cd 到给定目录
	\item 列出目录内容 ls -l
	\item 如果看到一个有趣的文件,用 file 命令确定文件内容
	\item 如果文件看起来像文本,试着用 less 命令浏览它
\end{enumerate}
\fboxrule=3pt \fboxsep=2pt
\begin{colorboxed}[boxcolor=lightgray,bgcolor=white]
\textbf{记得复制和粘贴技巧!}如果你正在使用鼠标,双击文件名,来复制它,然后按下鼠标中键,粘贴文件名到命令行中。
\end{colorboxed}

\par 在系统中游玩时,不要害怕粘花惹草。普通用户是很难把东西弄乱的。那是系统管理员的工作! 如果一个命令抱怨一些事情,不要管它,尽管去玩别的东西。花一些时间四处走走。 系统是我们自己的,尽情地探究吧。记住在 Linux 中,没有秘密存在! 表3-4仅仅列出了一些我们可以浏览的目录。闲暇时试试看!


%跨页表格

\begin{center} \tablecaption{Linux 系统中的目录 \label{tab:super}}
\tablefirsthead{%
\rowcolor[gray]{0.8}
\multicolumn{1}{l}{\textbf{目录}} & 
\multicolumn{1}{c}{\textbf{注解}} \\ }
\tablehead{\multicolumn{2}{c}{%
\small 表 \ref{tab:super} (续) } \\
\rowcolor[gray]{0.8}
\multicolumn{1}{l}{\textbf{目录}} & 
\multicolumn{1}{c}{\textbf{注解}} \\}
\tabletail{\bottomrule
\multicolumn{2}{c}{\small 接下页} \\
}
\tablelasttail{\bottomrule}

\begin{supertabular}{p{2.cm}p{13cm}}

/	& 根目录,万物起源。\\
\midrule
/bin & 包含系统启动和运行所必须的二进制程序。\\
\midrule
/boot & 
包含 Linux 内核,最初的 RMA 磁盘映像(系统启动时,由驱动程序所需),和 启动加载程序。
有趣的文件:
\begin{itemize} 
	\item  /boot/grub/grub.conf or menu.lst, 被用来配置启动加载程序。
	\item /boot/vmlinuz,Linux 内核。
\end{itemize} \\
\midrule
/dev & 这是一个包含设备结点的特殊目录。“一切都是文件”,也使用于设备。 在这个目录里,内核维护着它支持的设备。\\
\midrule /etc & 	
这个目录包含所有系统层面的配置文件。它也包含一系列的 shell 脚本, 在系统启动时,这些脚本会运行每个系统服务。这个目录中的任何文件应该是可读的文本文件。
有意思的文件:虽然/etc 目录中的任何文件都有趣,但这里只列出了一些我一直喜欢的文件:
\begin{itemize}
	\item /etc/crontab, 定义自动运行的任务。
	\item /etc/fstab,包含存储设备的列表,以及与他们相关的挂载点。
	\item /etc/passwd,包含用户帐号列表。
\end{itemize} \\
\midrule /home & 在通常的配置环境下,系统会在/home 下,给每个用户分配一个目录。普通只能 在他们自己的目录下创建文件。这个限制保护系统免受错误的用户活动破坏。\\
\midrule /lib & 包含核心系统程序所需的库文件。这些文件与 Windows 中的动态链接库相似。 \\
\midrule/lost+found	& 每个使用 Linux 文件系统的格式化分区或设备,例如 ext3文件系统, 都会有这个目录。当部分恢复一个损坏的文件系统时,会用到这个目录。除非文件系统 真正的损坏了,那么这个目录会是个空目录。 \\
\midrule/media & 在现在的 Linux 系统中,/media 目录会包含可移除媒体设备的挂载点, 例如 USB 驱动器,CD-ROMs 等等。这些设备连接到计算机之后,会自动地挂载到这个目录结点下。 \\
\midrule/mnt & 在早些的 Linux 系统中,/mnt 目录包含可移除设备的挂载点。\\
\midrule /opt & 这个/opt 目录被用来安装“可选的”软件。这个主要用来存储可能 安装在系统中的商业软件产品。\\
\midrule/proc & 这个/proc 目录很特殊。从存储在硬盘上的文件的意义上说,它不是真正的文件系统。 反而,它是一个由 Linux 内核维护的虚拟文件系统。它所包含的文件是内核的窥视孔。这些文件是可读的, 它们会告诉你内核是怎样监管计算机的。\\
\midrule /root & root 帐户的主目录。\\
\midrule /sbin & 这个目录包含“系统”二进制文件。它们是完成重大系统任务的程序,通常为超级用户保留。\\
\midrule /tmp & 这个/tmp 目录,是用来存储由各种程序创建的临时文件的地方。一些配置,导致系统每次 重新启动时,都会清空这个目录。\\
\midrule /usr & 在 Linux 系统中,/usr 目录可能是最大的一个。它包含普通用户所需要的所有程序和文件。\\
\midrule /usr/bin & /usr/bin 目录包含系统安装的可执行程序。通常,这个目录会包含许多程序。\\
\midrule /usr/lib & 包含由/usr/bin 目录中的程序所用的共享库。\\
\midrule /usr/local & 这个/usr/local 目录,是非系统发行版自带,却打算让系统使用的程序的安装目录。 通常,由源码编译的程序会安装在/usr/local/bin 目录下。新安装的 Linux 系统中,会存在这个目录, 但却是空目录,直到系统管理员放些东西到它里面。\\
\midrule /usr/sbin & 包含许多系统管理程序。\\
\midrule /usr/share & /usr/share 目录包含许多由/usr/bin 目录中的程序使用的共享数据。 其中包括像默认的配置文件,图标,桌面背景,音频文件等等。\\
\midrule /usr/share/doc & 大多数安装在系统中的软件包会包含一些文档。在/usr/share/doc 目录下, 我们可以找到按照软件包分类的文档。\\
\midrule /var & 除了/tmp 和/home 目录之外,相对来说,目前我们看到的目录是静态的,这是说, 它们的内容不会改变。/var 目录是可能需要改动的文件存储的地方。各种数据库,假脱机文件, 用户邮件等等,都驻扎在这里。\\
\midrule /var/log & 这个/var/log 目录包含日志文件,各种系统活动的记录。这些文件非常重要,并且 应该时时监测它们。其中最重要的一个文件是/var/log/messages。注意,为了系统安全,在一些系统中, 你必须是超级用户才能查看这些日志文件。\\

\end{supertabular}
\end{center}




% section 旅行指南 (end)



% \section{符号链接} % (fold)
\label{sec:符号链接}

在我们到处查看时,我们可能会看到一个目录,列出像这样的一条信息:
\begin{lstlisting}
lrwxrwxrwx 1 root root 11 2007-08-11 07:34 libc.so.6 -> libc-2.6.so 
\end{lstlisting}

\par 注意,这条信息第一个字符是``1'',并且看起来像有两个文件名? 这是一个特殊文件,叫做符号链接(也称为软链接或者 symlink)。 在大多数类似 Unix 系统中,有可能一个文件被多个文件名参考。虽然这种特性的意义并不明显,但它真地很有用。

\par 描绘一下这样的情景:一个程序要求使用某个包含在名为``foo''文件中的共享资源,但是``foo''经常改变版本号。 这样,在文件名中包含版本号,会是一个好主意,因此管理员或者其它相关方,会知道安装了哪个``foo''版本。 这又会导致一个问题。如果我们更改了共享资源的名字,那么我们必须跟踪每个可能使用了 这个共享资源的程序,当每次这个资源的新版本被安装后,都要让使用了它的程序去寻找新的资源名。 这听起来很没趣。

\par 这就是符号链接存在至今的原因。比方说,我们安装了文件``foo'' 的 2.6 版本,它的 文件名是 ``foo-2.6'',然后创建了叫做``foo'' 的符号链接,这个符号链接指向 ``foo-2.6''。 这意味着,当一个程序打开文件 ``foo''时,它实际上是打开文件 ``foo-2.6''。 现在,每个人都很高兴。依赖于``foo''文件的程序能找到这个文件,并且我们能知道安装了哪个文件版本。 当升级到 ``foo-2.7'' 版本的时候,仅添加这个文件到文件系统中,删除符号链接 “foo”, 创建一个指向新版本的符号链接。这不仅解决了版本升级问题,而且还允许在系统中保存两个不同的文件版本。 假想 ``foo-2.7''有个错误(该死的开发者!),那我们得回到原来的版本。 一样的操作,我们只需要删除指向新版本的符号链接,然后创建指向旧版本的符号链接就可以了。

\par 在上面列出的目录(来自于 Fedora 的 /lib 目录)展示了一个叫做 ``libc.so.6''的符号链接,这个符号链接指向一个 叫做 “libc-2.6.so” 的共享库文件。这意味着,寻找文件 ``libc.so.6''的程序,实际上得到是文件 ``libc-2.6.so''。 在下一章节,我们将学习如何建立符号链接。


% section 符号链接 (end)

% \section{硬链接} % (fold)
\label{sec:硬链接}

尽然在讨论链接问题,我们需要提一下,还有一种链接类型,叫做硬链接。硬链接同样允许文件有多个名字, 但是硬链接以不同的方法来创建多个文件名。在下一章中,我们会谈到更多符号链接与硬链接之间的差异问题。
% section 硬链接 (end)

% \section{拓展阅读} % (fold)
\label{sec:拓展阅读4}

\begin{itemize}
	\item 完整的 Linux 文件系统层次标准可通过以下链接找到:\\
	\url{http://www.pathname.com/fhs/}
	\item 一篇关于UNIX和类UNIX系统目录结构的文章:\\
	\url{http://en.wikipedia.org/wiki/Unix_directory_structure}
	\item ASCII 文本格式的详细描述: \\
	\url{http://en.wikipedia.org/wiki/ASCII}
\end{itemize}
% section 拓展阅读 (end)
