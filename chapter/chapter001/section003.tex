\section{试试运行一些简单命令}

现在,我们学习了怎样输入命令,那我们执行一些简单的命令吧。第一个命令是 date。 这个命令显示系统当前时间和日期。

\begin{lstlisting}
[me@linuxbox ~]$ date
Thu Oct 25 13:51:54 EDT 2007
\end{lstlisting}

\par 一个相关联的命令,cal,它默认显示当前月份的日历。

\begin{lstlisting}
[me@linuxbox ~]$ cal
October 2007
Su Mo Tu We Th Fr Sa
1 2 3 4 5 6
7 8 9 10 11 12 13
14 15 16 17 18 19 20
21 22 23 24 25 26 27
28 29 30 31
\end{lstlisting}

\par 查看磁盘剩余空间的数量,输入 df:
\begin{lstlisting}
[me@linuxbox ~]$ df
Filesystem           1K-blocks      Used Available Use% Mounted on
/dev/sda2             15115452   5012392   9949716  34% /
/dev/sda5             59631908  26545424  30008432  47% /home
/dev/sda1               147764     17370   122765   13% /boot
tmpfs                   256856         0   256856    0% /dev/shm
\end{lstlisting}

\par 同样地,显示空闲内存的数量,输入命令 free。

\begin{lstlisting}
[me@linuxbox ~]$ free
total       used       free     shared    buffers     cached
Mem:       2059676     846456    1213220          0
44028      360568
-/+ buffers/cache:     441860    1617816
Swap:      1042428          0    1042428
\end{lstlisting}

